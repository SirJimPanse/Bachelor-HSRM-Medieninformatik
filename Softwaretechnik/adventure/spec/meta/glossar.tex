\svnid{$Id: glossar.tex 95 2012-04-29 22:02:10Z dgens001 $}
\newglossaryentry{Spieler}{
	name={Spieler},
	plural={Spieler},
	description={Der Spieler in Form der Spielfigur, welche vom Anwender kontrolliert wird}
}

\newglossaryentry{Spiel}{
	name={Spiel},
	plural={Spiele},
	description={Die Interaktion mit der Anwendung nach dem Start eines neuen Spiels und dem 
		Erreichen des \gls{Spielziel}s}
}

\newglossaryentry{Spielziel}{
	name={Spielziel},
	plural={Spielziele},
	description={Um das \gls{Spiel} erfolgreich zu beenden, muss der \gls{Spieler} alle \glspl{Spielziel} 
		erreichen. \glspl{Spielziel} sind meist ein oder mehrere \glspl{Schein}}
}

\newglossaryentry{Savegame}{
	name={Savegame},
	plural={Savegames},
	description={Bezeichnet eine Datei, in der ein laufendes Spiel gesichert wird, 
		um es später fortzusetzen}
}

\newglossaryentry{Charakter}{
	name={Charakter},
	plural={Charaktere},	
	description={Das Alter Ego (seltener: Avatar) des Spielers {\it Ingame\/}}
}

\newglossaryentry{Dialog}{
	name={Dialog},
	plural={Dialoge},
	description={Meist eine Frage oder ein einfacher Satz mit (mehreren) möglichen Antworten, welche 
		auf einen weiteren Dialog verweisen oder das Ende der Unterhaltung markieren}
}

\newglossaryentry{Erwiderung}{
	name={Erwiderung},
	plural={Erwiderungen},
	description={Eine einfache Aussage oder Frage als Reaktion auf die eine NPC}
}

\newglossaryentry{GameObject}{
	name={GameObject},
	plural={GameObjects},
	description={Elemente im Spiel, z.B. \glspl{Charakter}, Gegenstände oder Scheine }
}

\newglossaryentry{Objekt}{
	name={Objekt},
	plural={Objekte},
	description={Elemente im Spiel, mit denen der \gls{Spieler} interagieren kann, die er jedoch nicht
		in sein \gls{Inventar} aufnehmen kann}
}

\newglossaryentry{Ingame}{
	name={Ingame},
	description={Bezeichnet den Zustand des Spiels, in dem gespielt werden kann. Übergeordnet auch 
		Ingamemodus genannt}
}

\newglossaryentry{Campus}{
	name={Campus},
	plural={Campi},
	description={Campus im Bezug auf das CampusAdventure. Synonym und Welt}
}

\newglossaryentry{Welt}{
	name={Spielwelt},
	description={Die Gesamtheit der virtuellen Umgebung bestehend aus dem Campus, Räumen, Items, 
		NPCs, Quests und Spieler}
}

\newglossaryentry{Gegenstand}{
	name={Gegenstand},
	plural={Gegenstände},
	description={Ein virtueller Gegenstand, der wenigstens eine Form der Interaktion mit dem Spieler zulässt}
}

\newglossaryentry{Automat}{
	name={Automat},
	plural={Automaten},
	description={Ein \gls{Gegenstand}, an dem der \gls{Spieler} evtl. \glspl{Item} erhält}
}

\newglossaryentry{Item}{
	name={Item},
	plural={Items},
	description={Ein \gls{Gegenstand}, der vom \gls{Spieler} in sein \gls{Inventar} aufgenommen werden kann}
}

\newglossaryentry{Inventar}{
	name={Inventar},
	description={Eine Art virtueller Rucksack mit beschränkter Kapazität,
	in dem z.B. {\it Items\/} aufgenommen werden können}
}

\newglossaryentry{Raum}{
	name={Raum},
	plural={Räume},
	description={Ein abgeschlossenes Areal innerhalb der Map. Räume müssen dem Spieler nicht zugänglich 
		sein. Räume bestehen aus {\it Feldern\/}}
}

\newglossaryentry{Eingang}{
	name={Eingang},
	plural={Eingänge},
	description={Ein möglicher Zugang zu einem {\it Raum\/}. Eingänge können verschlossen sein}
}

\newglossaryentry{Feld}{
	name={Feld},
	plural={Felder},
	description={Felder sind räumliche (ebene) Einheiten quadratischer Größe. Felder können sowohl 
		Items als auch NPCs oder it Quests enthalten. Der Spieler bewegt sich von Feld zu Feld}
}

\newglossaryentry{Semester}{
	name={Semester},
	description={Semester sind eine Abfolge von \glspl{Schein}n bzw. die Summe der durch die 
		     \glspl{Schein} verdienten \gls{cps} und bringen oft inhaltliche Neuerungen ins Spiel}
}

\newglossaryentry{Schein}{
	name={Schein},
	plural={Scheine},
	description={\glspl{Schein}  müssen vom \gls{Spieler} erlangt werden, um das \gls{Spielziel} zu erreichen. 
	\glspl{Schein} sind atomar und bringen dem \gls{Spieler} \gls{cps}}
}

\newglossaryentry{Bedingung}{
	name={Bedingung},
	plural={Bedingungen},
	description={Eine vom \gls{Spieler} zu erfüllende Forderung des Spiels. Bedingungen können erfüllt oder
		nicht erfüllt sein}
}

\newglossaryentry{Abschluss}{
	name={Abschluss},
	plural={Abschlüsse},
	description={Ein mögliches \gls{Spielziel} des \gls{Spieler}s. Ein \gls{Abschluss} hat \gls{Semester}}
}

\newglossaryentry{Menuemodus}{
	name={Menümodus},
	plural={Menümodi},
	description={Der Zustand der Applikation in dem nicht gespielt werden kann. Man kann aber in den
		Ingamemodus wechseln}
}

\newglossaryentry{Aktivitaetsbereich}{
	name={Aktivitätsbereich},
	plural={Aktivitätsbereiche},
	description={\gls{Ingame} wird das Fenster in vier dieser Bereiche unterteilt, von denen jeder einen
		anderen Aspekt des Spiels darstellt}
}

\newglossaryentry{Sichtbereich}{
	name={Sichtbereich},
	description={Einer der vier \glspl{Aktivitaetsbereich} des Spiels \gls{Ingame}. Hier wird die virtuelle
		Szene in einer Pseudo-3D-Perspektive dargestellt}
}

\newglossaryentry{Infobereich}{
	name={Infobereich},
	description={Einer der vier \glspl{Aktivitaetsbereich} des Spiels \gls{Ingame}. Hier wird das Inventar
		mit seinen Slots oder textuelle Information bei Interaktionen dargestellt}
}

\newglossaryentry{Handbereich}{
	name={Handbereich},
	description={Einer der vier \glspl{Aktivitaetsbereich} des Spiels \gls{Ingame}. Hier wird entweder eine
		leere Hand oder das momentan in der Hand gehaltene Objekt dargestellt}
}

\newglossaryentry{Statusbereich}{
	name={Statusbereich},
	description={Einer der vier \glspl{Aktivitaetsbereich} des Spiels \gls{Ingame}. Hier wird ein Portät des
		Charakters dargestellt}
}

\newacronym{npc}{NPC}{Non-Player-Character}
\newacronym{npcs}{NPCs}{Non-Player-Characters}
\newacronym{cp}{CP}{CreditPoint}
\newacronym{cps}{CPs}{CreditPoints}
\newacronym{ki}{KI}{Künstliche Intelligenz}
