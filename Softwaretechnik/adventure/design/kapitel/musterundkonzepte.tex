\svnid{$Id: musterundkonzepte.tex 180 2012-05-24 15:26:04Z dgens001 $}
\chapter{Muster und Konzepte}

\subsection{Listener-Konzept}
Die Anwendung verwendet \textit{PropertyChangeListener} und \textit{PropertyChangeSupport}.
Erstere werden von solchen Klassen verwendet, deren Objekte auf Ereignisse aus anderen Bereichen des \gls{Spiel}s warten.
Sie melden sich als Listener für bestimmte Attribute und Funktionen an. Ändern sich diese, erfahren die Listener
das und können im eigenen Objekt für Methodenaufrufe sorgen.
 
Für das Erfahren dieser Änderungen ist der \textit{PropertyChangeSupport} zuständig. 
Aus der Klasse heraus, in der er sich befindet, \textit{feuert} er \textit{PropertyChangeEvents}, 
die an die zugehörigen \textit{PropertyChangeListener} adressiert sind.

\subsection{MVC}
Für die Darstellung der Anwendung ist die Klasse \textit{Fenster} verantwortlich, unabhängig davon in welchem Modus 
- Menü oder Ingame - sich das \gls{Spiel} im Momemt befindet. Sie setzt sich wiederum aus verschiedenen Klassen, 
den Views, zusammen. Diese vereinen in sich jeweils die Informationen, die für die Darstellung der entsprechenden 
View von Nöten sind.

Die \textit{SpielerV} ist für die Anzeige des \gls{Statusbereich}s verantwortlich. Ihre Informationen kommen zum einen vom
\textit{Spieler} selbst und andererseits von \textit{NPC}s, \textit{Objekt}en oder \textit{Feld}ern mit denen aktuell interagiert wird.
\textit{Spieler} feuert Events, die den View anweist, die benötigten Informationen aus den entsprechenden Models zu lesen.
 
Beispiele für die Darstellung durch die \textit{SpielerV} sind \glspl{Dialog} und das Aufnehmen von \glspl{Item} von 
einem \textit{Feld} in die \textit{Hand} von \textit{Spieler}.

Bei der Klasse \textit{Sichtbereich} handelt es sich um die \gls{Welt} aus der Ego-Perspektive des \gls{Spieler}s.
Der gesamte \textit{Sichtbereich} setzt sich aus 32 \textit{JPanel}s zusammen, durch die sichergestellt wird, dass jede in der
Anwendung vorgesehene Kombination von Texturen darstellbar ist.

Es lassen sich zwei Unterbereiche abgrenzen. Das \textit{Feld} auf dem sich der \gls{Spieler} aktuell befindet und die beiden
in \gls{Blickrichtung} vor ihm liegenden \glspl{Feld} werden samt darauf befindlicher \glspl{Gegenstand} und \gls{npcs} dargestellt.
Alle anderen \glspl{Feld} und zugehörige Wände werden ausschließlich als Texturen angezeigt.

Unabhängig davon werden alle \glspl{Feld} einzeln durch eine separate \textit{FeldV} erstellt. Die Informationen hierzu 
kommen wiederum von den Models der einzelnen Felder(\textit{FeldM}).

Die letzte verbleibende View ist die \textit{MenuCV}, die gleichzeitig als Controller fungiert. 
Da das Menü nur eine grafische Oberfläche ohne abgrenzbare Logik ist, ist diese Zusammenführung sinnvoll.
Das Menü wird vor dem Start eines neuen oder Laden eines bereits begonennen \gls{Spiel}s als Hauptmenü angezeigt. 
Es dient gleichzeitig als Pausemenü, sollte das \gls{Spiel} zwischendurch unterbrochen werden.
In diesem Fall behalten die oben genannten Views ihre aktuellen Informationen, werden aber nicht angezeigt während
das Menü aktiv ist.

Die einzelnen Views erben jeweils von der Klasse \textit{JPanel}. Die Klasse \textit{Fenster}, die, wie oben erklärt die 
Views zur gesamten \gls{Anzeige} zusammensetzt, erbt von der Klasse \textit{JFrame}.
Die Views und das \textit{Fenster} implementieren Funktionen des o.g. Listener-Konzeptes, um ihre
Darstellungs-Informationen von den entsprechenden Models zu erhalten.

